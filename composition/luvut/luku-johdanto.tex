\chapter{Johdanto}

Tietojenkäsittelyssä käsitellään valtavia määriä dataa.

Tiedon esittämislle kompaktimmassa muodossa on laajalti
  tarvetta, milloin tallennuskapasiteetin säästämisen,
  milloin datan tehokkaan tunnistamiseen vuoksi.

Laskentakyvyn kasvaessa myös tietoliikenne ja tallennuskapasiteetti kehittyvät.

Rikolliset tahot hyödyntäväþ tietojenkäsittelyä [MISSÄ].

Tietojenkäsittelyn kehitys luo uusia, rikollisia intressejä tyydyttäviä mahdollisuuksia.

Kasvava datavirta helpottaa haittaohjelmien huomaamatonta
  leviämistä.

Sekä tekninen rikostutkinta että kyberrikollisuus
  ovat riippuvaisia tietotekniikan kehityksestä.
Innovaatiot luovat mahdollisuuksia kehittää
  rikollisia menetelmiä sekä toisaalta ratkaisuja
  haitallista toimintaa vastaan.

Tiivistystä voidaankin hyödyntää keinona haittaohjelmien havaitsemisessa suuresta datamäärästä.




