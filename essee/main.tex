\documentclass[12pt, a4paper]{article}

\usepackage[a4paper, top=3cm, bottom=4cm, left=2.6667cm, right=2.6667cm]{geometry}
\usepackage[T1]{fontenc}
\usepackage[utf8]{inputenc}
\usepackage{csquotes}
\usepackage[finnish]{babel}
\usepackage[final, babel=true, stretch=36]{microtype}
% \usepackage{palatino}

\usepackage{setspace}
\usepackage[style=apa]{biblatex}

\usepackage{amsmath}
\usepackage{amssymb}

\addbibresource{refs.bib}
\setstretch{1.25}

\title{Pätkäistyt hajautusarvot}
\author{Valtteri Varvikko}
\date{11.\ lokakuuta 2022}

\begin{document}
	\maketitle

	Tietojenkäsittelyssä käsitellään valtavia määriä dataa. Tiedon esittämiselle kompaktimmassa
	muodossa on laajalti tarvetta, milloin tallennuskapasiteetin säästämisen, milloin
	datan tehokkaan tunnistamiseen vuoksi. Tiivistää voidaan muun muassa pakkausalgoritmeilla ja
	hajautusfunktioilla. 
	Pakkauksesta poiketen hajautusfunktiot tiivistävät datan palauttamattomaan muotoon.
	Toisaalta hajautusarvo sisältää uniikin tunnisteen, jolla
	alkuperäinen data on tunnistettavissa. 
	Ihmissilmä löytää suurestikin poikkevista tiedostoista yhtäläisyyksiä,
	mutta monet hajautusfunktiot tuottavat pääosin aivan eri arvon.
	Samankaltaisuuksien tunnistamiseen kryptografisista tiiviistefunktioista ei ole,
	mutta niitä hyödyntävät \textit{fuzzy hash} -algoritmit mahdollistavat
	yhtenäisten rakenteiden havaitsemisen.

	\section*{Piecewise hashing}

	Kryptografiset hajautusfunktiot tuottavat samalla syötteellä saman tuloksen.
	Syötettä prosessoidessa jokainen alkio muuttaa generoitavan hajatusarvon
	edellisestä tunnistamattomaksi --- tämä ilmiö on olennainen kryptografisissa sovelluksissa
	ja siten tarkoituksellinen.
	Identtiset syötteet ovat todettavissa identtisiksi yhtäläisen hajautusarvon
	perusteella, mutta yhdenkin bitin muutos tuottaa edellisestä tunnnistamattoman
	hajautusarvon. Näin ollen samankaltaiset syötteet eivät ole tunnistettavissa
	samankaltaisiksi kryptografisilla hajautusfunktiolla.

	Syötteesä esiintyviä samankaltaisuuksia kutsutaan
	\textit{homologioiksi} \parencite{IDENT}. Homologioiden tunnnistaminen onnistuu hajautusarvoja
	vertaamalla, mutta tässä tapauksessa hajautusarvo on johdettava
	syötteessä esiintyvien homologioiden rajaamista osajonoista koko syötteen sijaan.
	Tätä tavoitetta ratkisemaan on laadittu kryptografisista hajautusfunktioista
	poikkeava algoritmityyppi. \textit{Fuzzy hash} -algoritmit tulkitsevat
	syötettä tarkastellen useampaa alkiota kerralla, joten alkion
	hajautusarvoon vaikuttaa myös naapurialkioiden arvo. Näin eroavaisuudet
	syötteissä aiheuttavat muutoksia hajautusarvoon ainoastaan paikallisesti,
	ja hajautusarvo säilyy muilta osin lähestulkoon muuttumattomana.

	Pätkäisty hajautusarvo voidaan toteuttaa esimerkiksi
	algoritmilla, joka jakaa syötteen $n$ alkion osajonoiksi.
	Alkio $a_i$, missä $n \mid i$, sekä sitä
	edeltävät alkiot $a_{i - 1}, a_{i - 2}, \ldots, a_{i - n}$
	redusoidaan esimerkiksi yhteenlaskulla skalaariksi, josta generoidaan
	tälle osajonolle hajautusarvo. Hajautusarvo lasketaan myös syötteen
	loppuun mahdollisesti jääville alle $n$ alkiolle. Generoidut hajautusarvot
	palautetaan jonona, jota voidaan vertailla
	muihin kuvatusti generoituihin hajautusarvoihin. Homologiat
	ilmenevät hajautusarvojonossa identtisinä hajautusarvoina; poikkeavat
	hajautusarvot taas viittaavat eroihin syötteiden välillä.

	\subsection*{Context triggered piecewise hashing}

	Edellä kuvattu ratkaisu toimii kohtalaisesti samanmittaisille
	syötteille, joissa alkioita on ainoastaan korvattu toisella
	kajoamatta syötteen pituuteen.
	Tulos kuitenkin häiriintyy pahasti, jos
	syötteen koko muuttuu alkuperäiseen verrattuna alkioita
	lisättäessä tai poistettaessa. Muiden alkioiden
	sijainti syötteessä vaihtuu, ja seuraavat osajonot sisältävät
	useimmiten alkuperäisestä poikkeavan alkiojonon.
	Muutokset ulottuvat syötteen loppuun asti, jolloin
	jokainen muuttunut osajono tuottaa alkuperäisestä
	poikkeavan hajautusarvon. Algoritmin palauttama hajautusarvojono
	saattaa merkittävästi poiketa alkuperäiselle syötteelle
	generoidusta hajautusarvostajonosta, ja sattumanvaraisen syötteen
	alusta poistettu alkio tuottanee kryptografisen hajautusfunktion
	tavoin aiemmasta täysin tunnistamattoman hajautusarvon.
	Hajautusarvo muuttuu sitä enemmän, mitä pienempään syötteen
	indeksiin alkion lisäys tai poisto kohdistuu.

	\textit{Context triggered piecewise hashing (CTPH)} on tekniikka
	\textit{fuzzy hash} -algoritmien toteuttamiseen. Välttääkseen edellä kuvatun ongelman
	CTPH-algoritmit jakavat syötteen vakiokokoisten osajonojen
	sijaan vaihtelevan kokoisiin osajonoihin, jotka määräytyvät dynaamisesti
	syötteen sisällön perusteella. Tällöin alkion lisäys tai poisto vaikuttaa
	ihanteellisesti yhden, ja enimmilläänkin vain lähimpien osajonojen
	hajautusarvoon eikä heijastu muihin osajonoihin.

	Monissa CTPH-algoritmeissa osiointi perustuu \textit{laukaisimiin (trigger)},
	joilla määritetään osajonon päätepiste alkion kontekstin perusteella \parencite{IDENT}. Laukaisin
	aktivoituu tietyllä syötteestä generoidulla arvolla ja ilmaisee
	määritellyn kokonaisuuden päättymistä. Käytännössä konteksti tarkoittaa
	alkion lähimmissä indekseissä olevia alkioita, joten kontekstille generoitu 
	hajautusarvo on riippuvinen useamman alkion yhdessä määrittämästä kokonaisuudesta.
	Etu \textit{piecewise hasheihin} on syötteen osiointi
	kontekstin perusteella vakiokokoisten osajonojen sijaan.

	Kontekstin tunnistaminen onnistuu \textit{rolling hash} -hajautusfunktiolla. Hajautusarvo
	lasketaan syötettä iteroidessa jokaiselle alkiolle rekursiivisesti huomioiden muut tietyn
	ehdon täyttävät alkiot. Usein tämä ehto tarkoittaa syötteen viimeisimpiä alkiota.
	Hajautusarvon $h$ generointia alkiolle $a_i$ sekä viimeisimmille $n$ alkiolle voidaan
	havainnollistaa hajautusfunktion $R$ rekursiivisena kutsuna 
	\[h = R(a_i, R(a_{i-1}, R(a_{i-2}, \ldots, R(a_{i - n + 1}, a_{i - n}) \ldots ))).\]
	Arvoa laskiessa ei käytännössä ole tarpeen laskea jokaiselle edelliselle alkiolle
	hajautusarvoa, koska viimeisin hajautusarvo on
	jo laskettu edeltävistä alkioista. Uusi hajautusarvo on laskettavissa
	hyödyntämällä viimeksi generoitua hajautusarvoa, josta poistetaan
	hajautuksesta riippuen vanhimman alkion vaikutus esimerkiksi loogisilla operaatioilla.

	Mikäli \textit{rolling hash} aktivoi laukaisimeen,
	hajautusarvojonoon lisätään vallitsevan osajonon
	alkioille generoitu hajautusarvo.
	Hajautusarvon generointiin käytetään tyypillisesti perinteistä
	hajautusfunktiota, kryptografisia (MD5, SHA-1 tms.) tai ei-kryptografisia (FNV tms.).
	Kuvattua menetelmää toistaen syötteestä muodostuu hajautusarvojono,
		joka sisältää syötteen osajonoista generoidut, toisistaan
		riippumattomat hajautusarvot.

	\subsection*{Toteutuksista}

	CTPH-toteutuksia on markkinoilla useita, kuten ssdeep, SDHASH ja mvHASH-B.
	Eri algoritmien toteutukset poikkeavat toisistaan, ja osa lähestyy
	ongelmaa erilaisesta näkökulmasta. Tästä huolimatta algoritmit
	jakavat yhteisen dilemman, kuinka löytää samankaltaisuuksia ja
	toistuvien rakenteita mistä tahansa annetusta syötteestä.

	\paragraph{spamsum}
	Eräs varhainen CTPH-hajautusfunktio, \textit{spamsum}, tunnistaa
	syötteissä esiintyviä samankaltaisuuksia. spamsum-algoritmin
	totetus perustuu CPTH-tekniikan ajatukseen syötteen käsittelystä konteksiriippuvaisesti
	ja seuraa pääpiirteissään edellä esitettyä kuvausta CTPH-algoritmin toiminnasta.
	Suorituksen alussa spamsum määrittää laukaisimen arvoksi \textit{block sizen},
	joka lasketaan syötteen perusteella.
	Syöttettä iteroitaessa \textit{rolling hashia} päivitetään FNV-tiivistefunktion
	alkiolle generoimalla tiivistellä \parencite{IDENT}. 

	Hajautusarvojen vertailu vaatii näiden generoinnissa käytetyn
	\textit{block sizen}. Koska \textit{block size} on riippuvainen syötteestä,
	se palautetaan yhdessä hajautusarvon kanssa. spamsum iteroi
	syötteen kahdesti, sekä \textit{block sizella} $b$ että $2b$.
	Näin saadaan parannettua hajautusarvojen vertailumahdollisuutta, ja
	hajautusarvoja $i$ ja $j$ vertaillessa on täytyttävä joko
	$b_i = b_j$, $2b_i = b_j$ tai $b_i = 2b_j$ \parencite{IDENT}. Hajautusarvoa $i$ voi
	siis verrata toiseen hajautusarvoon, jonka \textit{block size} on joko $b_j$ tai kahden potenssi.

	\paragraph{ssdeep}
	spamsum-algoritmin pohjalta on kehitetty ssdeep-ohjelmisto
	konteksiriippuvaisten hajautusarvojen käsittelyyn \parencite{SSDEEP}. ssdeepin
	hajautusarvojen generointi toimii jokseenkin samoin kuin
	spamsumin. ssdeep generoi syötteestä konteksiriippuvaisen
	hajautusarvon. spamsumin tavoin
	hajautusarvo koostuu peräkkäisistä osajonoille generoiduista
	hajautusarvoista sekä käytetystä \textit{block sizesta}.

	ssdeepiä käytetään laajasti monenlaisiin käyttötarkoituksiin
	ollen alansa de facto -standardi \parencite{SSDEEP}.
	Eräs keskeinen alue on tietoturva, jossa ssdeepiä
	sekä vastaavia ohjelmistoja voidan käyttää tunnistamaan
	haittaohjelmia. Tyypillisesti yksittäisen hyökkääjätahon
	menettelyssä esiintyy kaavamaisia piirteitä, ja hyökkääjä
	saattaa esimerkiksi käyttää paikoitellen samaa koodia eri
	ohjelmissa \parencite{RANSO}. ssdeep voi tunnistaa muita, saman tahon laatimia
	haittaohjelmia yhdestä, haittaohjelmaksi todetun ohjelman
	lähdekoodista generoidusta CTPH:sta vertaamalla sitä
	em.\ epäiltyjen haittaohjelmien hajautusarvoihin.

	\paragraph{Heikkoudet}
	\textit{Fuzzy hash} -algoritmien havaitaan suoriutuvan kiitettävästi \parencite{RANSO}.
	Erityisesti ssdeep-funktion mainitaan kykenevän yhdistämään laajastikin
	muuttettuja sekä puuttellisia tiedostoja alkuperäisiin. \textit{Fuzzy hashing} ei
	kuitenkaan ole aukoton tekniikka. Nämä algoritmit ovat vaikeasti
	ennustettavia, eikä syötteiden samankaltaisuudelle ole johdettavissa
	yleistä raja-arvoa, jonka alittaessaan syötteet eivät ole enää samankaltaisia.
	Samankaltaisuus tietyssä tilanteessa ei ole laskettavissa oleva ongelma,
	vaan ohjelman käyttäjän näkemys perusten vallitseviin olosuhteisiin.
	Tulkintaa vaikeuttaa myös algoritmien syntaksiin perustuva analyysi
	--- hajautusarvojen vertailu ei kerro syötteen semanttisesta kontekstista.

	Datan pakkaus aiheuttaa merkittäviä haasteita \textit{fuzzy hash}
	-hajautusfunktioille \parencite{RANSO}. Ne eivät juurikaan kykene käsittelemään
	samankaltaisesta datasta generoituja hajautusarvoja.
	Eräs keskeinen \textit{fuzzy hashien} käyttökohde
	on tunnistaa haittaohjelmia, joiden levitys keskittyy
	merkittävissä määrin juuri tietoliikenteeseen jonka pakatun datan
	analysoinnin automatisointi ei onnistune.

	\paragraph{}
	\textit{Fuzzy hashing} ei ole innovaationa tuoreimmasta päästä, mutta sittemmin
	sille on luotu monenlaisia käyttökohteita. Enenevissä määrin tarvitaan
	eksaktin yhtäläisyyden lisäksi samankaltaisuuksien tunnistamista ja
	luokittelua. CTPH-algoritmeilla voidaan 
	vähentää toisteisuutta,
	tunnistaa haittaohjelmia tai
	löytää puutteellisen tiedoston alkuperäinen kappale rikostutkinnassa.
	Huomioitavaa on,
	että tulokset ovat numeerisia parametrejä,
	joista ei voi tilannesidonnaisuuden vuoksi johtaa yleispätevää mallia.
	Algoritmi tunnistaa ainoastaan homologiat,
	ihmisen tulee edelleen päättää, mitä saadusta tuloksesta
	kussakin tilanteesssa seuraa.
	\printbibliography{}
\end{document}

