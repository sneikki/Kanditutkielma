% \begin{abstract}{finnish}

% Tämä dokumentti on tarkoitettu Helsingin yliopiston tietojenkäsittelytieteen osaston opin\-näyt\-teiden ja harjoitustöiden ulkoasun ohjeeksi ja mallipohjaksi. Ohje soveltuu kanditutkielmiin, ohjelmistotuotantoprojekteihin, seminaareihin ja maisterintutkielmiin. Tämän ohjeen lisäksi on seurattava niitä ohjeita, jotka opastavat valitsemaan kuhunkin osioon tieteellisesti kiinnostavaa, syvällisesti pohdittua sisältöä.


% Työn aihe luokitellaan  
% ACM Computing Classification System (CCS) mukaisesti, 
% ks.\ \url{https://dl.acm.org/ccs}. 
% Käytä muutamaa termipolkua (1--3), jotka alkavat juuritermistä ja joissa polun tarkentuvat luokat erotetaan toisistaan oikealle osoittavalla nuolella.

% \end{abstract}

\begin{otherlanguage}{english}
\begin{abstract}

Tiivistys on yleisesti käytetty menetelmä haittaohjelmien tunnistamiseen.
Erityisesti moderneista haittaohjelmmista levitetään tyypillisesti
muunneltuja versioita, joita on vaikeampi tunnistaa. Kryptografisilla
tiivisteillä ei voida tunnistaa tällaisia samankaltaisia ohjelmia.
Samankaltaisuuksinen tunnistamiseen on kehitetty erilaisia sumeita tiivisteitä,
jotka kykenevät yhdistämään suurestikin poikkeavia syötteitä toisiinsa.
Sumeita tiivistetiä on paljon, ja paikoitellen ne eroavat toisistaan
merkittävästi. Tässä tutkielmassa käsitellään sumeiden tiivisteiden
soveltamista haittaohjelmien tunnistamiseen niiden
sisältämien samankaltaisuuksien perusteella. Erityisesti analysoidaan
eri ohjelmistojen soveltuvuutta ja puutteita sekä sumean tiivistyksen
haavoittuvuuksia.

\end{abstract}
\end{otherlanguage}
