\chapter{Sumea tiiviste\label{fuzzy-hash}}

% 4

Kryptografisia tiivisteitä käytetään laajasti datan tunnistamiseen \parencite{kornblum06}. Ne tuottavat
samalla syötteellä saman tiivisteen, ja data voidaan lähes varmuudella tunnistaa
samaksi yhtäläisen tiivisteen perusteella. Pieninkin muutos syötteessä tuottaa
kuitenkin aiemmasta tunnistamattoman tiivisteen. Vertailu kryptografisilla
tiivisteillä ei ole soveltuva keino samankaltaisten syötteiden tunnistamiseen
\citep{kornblum06}.

Samankaltaisuuden tunnistamista varten kehitetyillä sumeilla tiivisteillä
voidaan vaivatta tunnistaa toisistaan eroavia samankaltaisia syötteitä \parencite{kornblum06}. Sumeat
tiivisteet mahdollistavat suurestikin muutettujen tai puutteellisten tiedostojen
yhdistämisen alkuperäiseen versioon. Samanakaltaisuuksien tunnistamisella
on merkitystä myös haittaohjelma-analyysissa. Haittaohjelmia
kehittävillä tahoilla on taipumus toistaa kaavamaisia piirteitä,
joten yhden tunnistetun haittaohjelman avulla voidaan mahdollisesti
havaita muitakin haittaohjelmia \parencite{naik19}.

Tässä tutkielmassa sumeat tiivisteet jaetaan toimintaperiaatteen
mukaan luokkiin perustuen luokitukseen, jonka \textcite{martin-perez21}
esittelevät.
Tiivisteet on jaettu ominaisuuksiensa perusteella kolmeen luokkaan.
Seuraavassa aliluvussa käsitellään sumeiden tiivistefunktioiden
toimintaa yleisellä tasolla. Viimeisissä aliluvuissa käsitellään
kaksi sumeiden tiivisteiden luokkaa: tiivistetyt ominaisuusjonot sekä
tavujonot.

\section{Tiivistäminen ja vertailu}

Samankaltaisuuksia tunnistavia algoritmeja laatiessa keskitytään
koko syötteen tarkastelun sijaan pieniin osioihin.
Olennaista ei ole tunnistaa täsmälleen tietynlaista syötettä,
vaan löytää syötteestä vertailuun sopivia säännönmukaisuuksia. Näitä
rakenteita kutstuaan usein ominaisuuksiksi \parencite{martin-perez21}.
Ominaisuuksien valintakriteerit ja vertauslogiikka riippuvat
käytetystä algoritmista.

Monet algoritmit muodostavat sumean tiivisteen jonosta, joka
sisältää syötteestä valitut ominaisuudet tiivistettynä \parencite{martin-perez21}.
Osa tiivisteistä tiivistää koko syötteen, toisaalta
osa tiivistää ainoastaan osan. Huomionarvoista
on, että ominaisuuksien tiivistämiseen käytetään
sumeisssakin tiivisteissä kryptografisia tiivisteitä \parencite{kornblum06}.
Datan samankaltaisuuden vertailu sumeilla tiivisteillä etenee
vertaillen tiivistettyjä ominaisuuksia koko tiivisteen sijaan.
Samankaltaisuuden mittarina käytettävä ominaisuuksien
vertailufunktio riippuu käytetystä algoritmista.

\section{Tiivistetyt ominaisuusjonot}

Ominaisuusjonojen tiivistäminen on sumeiden tiivisteiden luokka,
johon kuuluu useita laajalti käytettyjä tiivisteitä. \textcite{martin-perez21}
esittävät tiivistetyt ominaisuusjonot yhtenä kolmesta tiivisteluokasta.
Tässä lähestymistavassa tarkastellaan ominaisuuksista
muodostuvia jonoja, eli syötteestä etsitään ominaisuuksia,
jotka tiivistetään jonoksi. Piirteen määrittely sekä
valinta on algoritmikohtaista.

\begin{table}[h]
   \centering
   \caption{Merkkijonolle luotu tiiviste, joka on määritelty paloittain.}
   \begin{tabular}{|c|c|c|c|c|c|c|c|c|c|c|c|} 
      \hline
      h & e & i                  &  & m & a                & a & i & l               & m & a & !                \\ 
      \hline
      \multicolumn{3}{|c|}{7b}   & \multicolumn{3}{c|}{7f} & \multicolumn{3}{c|}{e5} & \multicolumn{3}{c|}{53}  \\
      \hline
   \end{tabular}
\end{table}

Tiivistetty ominaisuusjono voidaan muodostaa määrittelemällä
tiiviste paloittain \parencite{martin-perez21}. Syöte jaetaan lohkoihin,
jolloin suuresta syötteestä muodostuu jono vertailtavia ominaisuuksia. 
Yksinkertaisessa paloittaisessa määrittelyssä syöte jaetaan
vakiokokoisiin lohkoihin, jotka kuvaavat syötteen ominaisuuksia
\parencite{kornblum06}.
Paloittain määritelty tiiviste muodostuu siis jonosta tiivistettyjä
lohkoja $H_i = R(a_j, a_{j - 1}, a_{j - 2}, \dots, a_{j - n})$,
missä $n$ käytetty lohkokoko ja $j = in$. Menetelmää havainnollistetaan
taulukossa 2.1, jossa syötteen lohkoille generoidaan tiivisteet
CRC-8-tiivistefunktiolla lohkokoon ollessa $n = 3$.

\begin{table}[h]
   \centering
   \caption{Paloittain määritelty tiiviste merkkijonolle, jossa merkkejä
      on vaihdettu.}
   \begin{tabular}{|c|c|c|c|c|c|c|c|c|c|c|c|} 
      \hline
      \textcolor{red}{h} & e & i                  &  & m & a                & a & i & l               & m & a & \textcolor{red}{?}                \\ 
      \hline
      \multicolumn{3}{|c|}{\textcolor{red}{38}}   & \multicolumn{3}{c|}{7f} & \multicolumn{3}{c|}{e5} & \multicolumn{3}{c|}{\textcolor{red}{09}}  \\
      \hline
   \end{tabular}
\end{table}

Taulukossa 2.2 on kuvattu muutoksia, jotka aiheutuvat, kun syötteen
alkioita vaihdetaan. Koska syötteen pituus säilyy muuttumattomana, muutokset
heijastuvat ainoastaan lohkoihin, joissa muutoksia on tapahtunut; keskimmäisten
lohkojen tiivisteet eivät ole muuttuneet lainkaan. Sen sijaan taulukossa 2.3
huomataan algoritmin toiminnan häiriintyvän pahasti, kun syötteen pituus muuttuu.
Alkion lisäys tai poisto muuttaa kyseistä alkiota seuraavien alkoiden indeksiä.
Syötteen alkuun lisätty alkio johtaa pahimmillaan kaikkien seuraavien lohkojen
tiivisteen muuttumisen tunnistamattomaksi.

\begin{table}[h]
   \centering
   \caption{Paloittain määritelty tiiviste johon on lisätty merkki.}
   \begin{tabular}{|c|c|c|c|c|c|c|c|c|c|c|c|c|} 
      \hline
      H & e & i                  & \textcolor{red}{,} &  & m                & a & a & i                                & l & m & a                                & !                    \\ 
      \hline
      \multicolumn{3}{|c|}{7b}   & \multicolumn{3}{c|}{\textcolor{red}{e9}} & \multicolumn{3}{c|}{\textcolor{red}{54}} & \multicolumn{3}{c|}{\textcolor{red}{03}} & \textcolor{red}{e7}  \\
      \hline
   \end{tabular}
\end{table}

Vakiokokoisilla lohkoilla esiintyvä ongelma ratkeaa, kun lohko sisältää saman
alkiojonon lisäyksistä riippumatta \parencite{kornblum06}. Kontekstiriippuvaisessa paloittaisessa tiivistyksessä
lohkot määritetään syötteen perusteella. Algoritmin toimintaperiaatteena on jakaa syöte
lohkoiksi siten, että lohkot sisältävät algoritmille määritellyn kontekstin.
Kontekstiriippuva paloittain määritelty tiiviste ei ole herkkä syötteen pituuden
muuttumisel le. Lisäämistä ja poistamisesta vaikuttaa ihanteellisesti yhden,
ja enimmilläänkin vain lähimpien naapurilohkojen tiivisteisiin eikä heijastu muihin lohkoihin.

Paloittaisen määritettelyn lisäksi sumeaa tiivistystä on lähestytty muilla tavoilla.
lohkoihin jakamisen sijaan syötteeestä voidaan etsiä tilastollisesti poikkeavia
piirteitä. Poikkeavia piirteitä havainnoiva algoritmi valitsee tietyn määrän
poikkeavimpia piirteitä ja tiivistää ne \citep{roussev10}. Varsinainen sumea tiiviste muodostuu
valittujen piirteiden tiivisteistä. Paloittain määriteltyjen tiivisteiden tavoin
tiivistämiseen käytetään kryptografista tiivistefunktiota. Huomionarvoista on,
että yleensä tiivistettäväksi valitut ominaisuudet kattavat syötteestä ainoastaan osan \citep{martin-perez21}.
Syötteestä ei siis tiivistetä kuin osa, toisin kuin paloittain määritteltäessä.

Poikkeavuudet säilyttävän tiivisteen lähestymistavassa tulkitaan,
että samalla tavalla merkittävästi poikkeavat syötteet liittyvät
toisiinsa. Ei siis ole todennäköistä, että satunnaisissa syötteissä
esiintyisi suuri määrä samoja, epätyypillisiä piirteitä. Kuvatussa
tilanteessa on pääteltävissä syötteiden olevan mitä luultavimmin
peräisin samasta alkuperästä, ja piirteet ovat muokkausten jälkeen
jääneet yhdistäväksi tekijäksi. Tällöin voidaan tehdä johtopäätös
ja olettaa syötteiden olevan tietyllä todennäköisyydellä
kokonaisuutenakin samankaltaisia.

Poikkeavuuksiin perustuvuvan algoritmin on etsittävä
syötteestä piirteet ja valittava niistä epätyypillisimmät.
Keskeinen ongelma onkin piirteen määrittely. Onglemaan
ei ole yksikäsitteistä ratkaisua, ja määritelmä
on tapauskohtaista. Esimerkiksi seuraavassa luvussa
käsiteltävässä Sdhash-ohjelmistossa piirre on määritelty
64 tavun merkkijonoksi \citep{roussev10}.
% ^ cyber threat hunting
% Breitinger et al.(Breitinger & Baier, 2012b)showed some weaknesses in sdhash and pre-sented improvements to the scheme | security analysis
% However, its strong suit is the detection of fragments and not comparison of files [p.8, breitinger12].


\section{Tavujono}

Tavujonosyötteet muodostavat tiivisteen jakamalla syötteen lohkoiksi,
joita verrataan samankaltaisuuden perusteella yhtäsuuruuden sijaan
\parencite{martin-perez21}.
Lohkoista rakennetaan syötteelle uusi esitystapa, joka sisältää
yksinkertaistetun kuvauksen syötteestä. Muutokset alkuperäisessä
syötteessä eivät vaikuta tietyissä rajoissa yksinkertaistettuun
muotoon, jonka perusteella samankaltaisuudet voidaan edelleen
määrittää. Tavujonotiivisteet eivät etsi syötteestä tietyntyyppisiä
ominaisuuksia kuten ominaisuustiivisteet.

Tavujonotiivisteille on määriteltävä logiikka, jolla lohkonta
suoritetaan. Lohkot voivat olla myös vakiokokoisia, jos niiden
tiivistykseen ei käytetä kryptografista tiivistettä; esimerkiksi
Mvhash-b-ohjelmistossa lohkot ovat vakiokokoisia. Lohkon arvoksi
asetetaan joko 0 tai 255, riippuen alkioiden enemmistöstä.
Enemmistöperiaatteen sekä arvojen pienen määrän vuoksi lohkosta
toiseen siirtyvien alkioiden vaikutus häipyy.

