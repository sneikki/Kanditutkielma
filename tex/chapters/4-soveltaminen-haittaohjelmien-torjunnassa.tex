\chapter{Soveltaminen haittaohjelmien torjunnassa\label{fuzzy-application}}

% 4

\section{Haittaohjelmien tunnistaminen}
Haittaohjelmien torjunta on laaja toimintakenttä, joka pitää
sisällääñ toisistaan poikkeavia osa-alueita. Haittaohjelmien
tunnistaminen on torjuntatyön keskeinen osa-alue, joka
sisältää myös laajan skaalan erilaisia lähestymistapoja.
\textcite{aslan20} esittävät haittaohjelma-analyysin
jaon staattiseeen sekä dynaamiseen analyysiin. Staattisessa analyysissa
tutkitaan haittaohjelmaa suorittamatta sen ohjelmakoodia lainkaan.
Dynaamisessa analyysissa keskitytään sen sijaan haittaohjelman 
käyttäytymisen tutkimiseen ohjelmakoodia suorittamalla.

Tiivistämisellä on merkitystä lähinnä staattisessa analyysissa.
Haittaohjelmien tunnistuksessa on tyypillisesti käytetty menetelmänä
haittaohjelmista luotujen tiivisteiden vertailua \parencite{sarantinos16}.
Tunnettujen haittaohjelmien tiivisteistä pidetään yllä tietokantaa.
Mikäli epäily haittaohjelmasta ilmenee, epäilty ohjelma tiivistetään
ja sitä verrataan tietokannassa oleviin tiivisteisiin. Yhtäläinen tiiviste
viittaa epäíllyn ohjelman olevan lähes varmuudella tunnettu haittaohjelma.
Haittaohjelmaksi todettu ohjelma voidaan nyt eliminoida ilman tarvetta
ohjelman suorituksenaikaisen toiminnan tarkastellulle dynaamisen
analyysin keinoin.

Perinteisesti haittaohjelmien tiivistämiseen on käytetty kryptografisia
tiivisteitä, esimerkkeinä MD5 ja SHA-1 \parencite{sarantinos16}. Kryptografiset
tiivisteet ovat suorituskykyisiä, mutta niillä ei voida tunnistaa muunneltuja
haittaohjelmia. \textcite{roussev11} mainitsemia kryptografisten tiivisteiden
rajoitteita ovat

\begin{itemize}
   \item Datan havaitseminen toisen datan sisältä
   \item Lähdekoodiversioiden havaitseminen
   \item Samaa alkuperää olevien dokumenttien tunnistaminen
\end{itemize}

On havaittu, että haittaohjelmia laativien tahojen toiminnassa
esiintyy toistuvia kaavamaisuuksia. Ilmiö heijastuu levitettävissä
haittaohjelmissa, joista löytyy yhteneviä piirteitä. Haittaohjelman
valmistaja saattaa hyödyntää samaa tai samankaltaista lähdekoodia
useassa haittaohjelmassa, ja tällaisten yhtäläisyyksien havaitseminen
rajaa potentiaalisten haittaohjelmien joukkoa pienemmäksi.

\section{Edellytykset}

Ohjelmistojen toteutukset poikkeavat toisistaan, mikä aiheuttaa
eroavaisuuksia tehtävistä suoriutumisessa. 
Ohjelmistoja voidaan luokitella sen perusteella, kuinka
ne määrittelevät samankaltaisuuden. Erilaisilla tavoilla
tunnistaa samankaltaisuuksia on vahvuutensa sekä heikkoutensa.
Tiettyyn tarkoitukseen optimaalisen ohjelmiston valinta
on tilannesidonnainen, ja toiset ohjelmistot suoriutuvat
paremmin tietyissä tilanteissa kuin toiset \parencite{pagani18}.

Tyypillisesti ohjelmistojen eroja on havainnoitu lähinnä kokeellisesti
\parencite{martin-perez21}.
Ohjelmistojen lähestymistapojen suuri määrä vaikeuttaa sumeiden tiivisteiden
vertailua. Sumeisiin tiivisteisiin liittyvän terminologian suhteen ei ole
vallitsevaa konsensusta, eikä universaalia vertailua mahdollistavaa luokittelua
ole ollut. \textcite{martin-perez21} esittävät luokittelun, joka mahdollistaa
muun muassa suorituskykyä, tulosten luotettavuutta sekä hyökkäysalttiutta.
tarkastelun.

\textcite{tlsh} esittää tekijöitä, jotka on syytä ottaa huomioon sumean tiivistämisen ohjelmistoa valitessa.

\begin{itemize}
\item Samanakaltaisuuden havaitsemiskyky
\item Kyky vastustaa hyökkäyksiä
\item Tiivisteen koko ja sen luomiseen kuluva aika 
\item Skaalautuus
\end{itemize}

Tarkkuus, jolla samankaltaisuuksia havaitaaan, on
tärkeä sumean tiivisteen käyttökelpoisuutta kuvaava
mittari. Tarkkuuteen vaikuttaaa toteutusperiaatteen
lisäksi syötteenä annetun datan tyyppi. Esimerkiksi
konteksiriippuvaisia paloittain määriteltyjä tiivisteitä,
kuten ssdeep, ei pidetä soveltuvina lähdekoodianalyysiin
\parencite{pagani18}. Epätodennäköisiin ominaisuuksiin
perustuvat tiivisteet, kuten Sdhash, sopivat tällaiseen tarkoitukseen
paremmin. Haittaohjelma-analyysissa keskeisessä roolissa on
konekielisen ohjelmankoodin analysointi, joten valitun ohjelmiston
vaikutus on suuri. Myös \textcite{roussev11} käsittelee Sdhashin
tarkkuttaa, minkä perusteella Sdhashin kyky tunnistaa haittaohjelmia
on ylivoimainen Ssdeepiin nähden. Ssdeepin, Sdhasin ja
Mvhash-b:n keskinäistä tarkkuuseroa
tutkivat \textcite{naik19} käyttäen WannaCry- ja
WannaCryptor-haittaohjelmia. Kaikkien kolmen 
todetaan suoriutuvan tehokkaasti tämäntyyppisten
haittaohjelmien tunnistamisesta.

\textcite{martin-perez21} esittää tiivisteiden toimintaan
vaikuttaviksi tekijöiksi syötteen kattavuuden sekä
tiivisteen koon. Ssdeepin ja Mvhash-b:n tiivisteet
perustuvat koko syötteeseen, eivätkä muutokset syötteessä
jää huomaamatta. Sdhashissa vain pieni osa syötteestä tiivistetään.
Osittainen syötekattavuus mahdollistaa hyökkäyksiä, jotka eivät
täydellä syötekattavuudella onnistu \parencite{martin-perez21}.

\section{Haittaohjelmien tunnistamisen välttäminen}

\textcite{pagani18} tutkivat Ssdeep-, Sdhash-, Tlsh- ja mrsh-v2-ohjelmistojen
suoriutumista konekielisen ohjelmakoodin yhtäläisyyksien löytämisessä.
Kolmessa tilanteessa analysoitiin kirjastojen tunnistamista,
erilaisten käännöskonfiguraatoiden vaikutusta sekä ohjelmien samankaltaisuutta.

Kirjastojen tunnistamisessa on kyse sisällytetyn datan havaitsemisesta
suuremmasta datamäärästä. Havaintona esitetään, ettei yksikään
ohjelmisto kyennyt yhdistämään kaikkia konekielisten objektitiedostojen
sisältämiä kirjastoja kokonaisiin konekielisiin ohjelmiin. Eniten
kirjastoja tunnisti Sdhash. Erityisesti heikoiten suorituneen Ssdeepin
mainitaan tuottaneen kaikissa tapauksissa pienimmän samankaltaisuusarvon.

Käännöskonfiguraatioiden kohdalla tutkittiin C-kääntäjän (gcc)
optimointitasojen sekä eri kääntäjien käytön vaikutusta. Molemmissa
tapauksissa Ssdeep suoriutui heikoimmin Sdhashin suoriutuessa
paremmin. Ohjelmien samankaltaisuuden tunnistusta vertaillessa
ohjelmakoodiin lisättiin sattumanvaraisesti tyhjiä konekäskyjä
sekä vaihdettiin sattumanvaraisia konekäksyjä. Havaittiin,
että jo muutamat muutokset tuottivat ajoittain käännettyyn
ohjelmaan, jota ohjelmistot eivät tunnistaneet. Ilmiöstä
todetaan, että pienet muutokset voivat saada aikaan
muutoksia myös muualla ohjelmakoodissa johtaen
tunnistamattomiin muutoksiin.

Haittaohjelman tunnistamista voidaan vaikeuttaa monin tavoin.
\textcite{aslan20} mainistevat useita menetelmiä, joilla haittaohjelman
tunnistamista voidaan vaikeuttaa. Menetelmiä, jotka aiheuttavat haasteita
sumeiden tiivistefunktioiden käytölle tunnistamisessa, ovat muun muassa
lähdekoodin salaus, metamorfoosi, naamiointi ja ohjelmiston pakkaus eri tavoin.
Kaikissa menetelmissä pyrkimysksenä on minimoida tiivistämisen hyöty
staattisessa haittaohjelma-analyysissa. Metamorfisena toteutettu haittaohjelma
muokkaa omia konekäskyjään pyrkien todellisuudessa käyttäytymiseen,
joka ei suoraan ilmene lähdekoodista staattisen analyysin keinoin \parencite{aslan20}.

Dataa voidaan naamioida semanttisen merkityksen muuttumatta siten, ettei se
kuitenkaan vaikuta samanlaiselta \parencite{oliver14}. Sumeiden tiivisteiden
tuloksi pyritään vääristämään naamioimalla dataa eri tavoin.
Oliver, Forman ja Cheng, 2014 analysoi Ssdeep-, Sdhash- ja Tlsh-ohjelmistojen herkkyyttä muutoksille. Ohjelmakoodin muutosten analysoinnissa pyrittiin muokkaamaan lähdekooditiedostoja siten, että suoritettava ohjelma on ekvivalentti ja tuottaa tiivisteen, jonka perusteella se ei ole tulkittavissa samankaltaiseksi alkuperäisen kanssa. Tutkimuksessa tutkittiin keinoja, joilla konekielisestä ohjelmakoodista luotua tiivistettä voidaan
muuttaa mahdollisimman paljon. Esitettyjä keinoja haittaohjelman naamioimiseksi
ovat muun muassa

\begin{itemize}
   \item Loogisten JA- ja TAI-operaation järjestäminen uudelleen
   \item Uusien kokonaisluku- ja merkkijonomuuttujien lisääminen
   \item Funktioiden järjestyksen muuttaminen
   \item Tyhjien käskyjen lisääminen
   \item Satunnaisen binäärimuotoisen datan lisääminen
   \item Merkkijonojen katkaiseminen
\end{itemize}

Tutkituista ohjelmstoista Ssdeep ja Sdhash eivät tunnistaneet kaikkia muunneltuja ohjelmia, mutta Tlsh kykeni tunnistamaan kaikki. Erityisesti muuttujien lisääminen laski Ssdeep-ohjelmiston tuottamaan vähäisintä yhtäläisyyttä kuvaavan arvon. Muutosten aiheuttamat erot tiivisteissä vaihtelevat menetelmittäin, ja eri ohjelmiston herkkyys muutoksille vaihtelee. Tulosten perusteella Tlsh-ohjelmiston esitetään olevaan sekä Ssdeep- ja Sdhash-ohjelmistoja luotettavampi ja turvallisempi vaihtoehto haittaohjelma-analyysiin.
