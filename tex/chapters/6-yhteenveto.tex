\chapter{Yhteenveto\label{summary}}
Samankaltaisuuden
tunnistamiseen on ollut tarvetta monessa paikassa,
ja aiemmat ratkaisut eivät ole tähän kyenneet. Erityisesti
tarvetta on haittaohjelmien tunnistamisessa, jossa on perinteisesti
käytetty kryptografisia tiivisteitä. Ongelmaa lähestyvät
sumeat tiivisteet mahdollistavat samankaltaisten haittaohjelmien
tunnistamisen.

Sumea tiivistys on uudehko tekniikka, mutta sumean tiivistyksen
ohjelmistoja on kehitetty laajasti. Eri ohjelmistot lähestyvät
samankaltaisuuden tunnistamista poikkeavin tavoin, mikä on tehnyt
vertailun ja luokittelun haastavaksi. Ohjelmistoista käsiteltiin
tunnetuimpiin kuuluvat ominaisuusjonoja tiivistäväþ Ssdeep ja Sdhash
sekä hieman uudempi Mvhash-b, joka keskittyy tavujonoihin.

Kaikki kolme käsiteltyä sumean tiivistyksen ohjelmistoa poikkeavat
toteutustavaltaan radikaalisti toisistaan. Poikkeavat esitystavat
samankaltaisuudelle vaikeuttavat ohjelmiston valintaa. Toiset
ovat päällekkäisiä, toiset taas aivan erilaisia. Ohjelmistojen
soveltuvuutta haittaohjelma-analyysin tarpeisiin pohdittiin
perinteisten haittaohjelmien tunnistamiskeinojen ja torjuntatyön
edellytysten pohjalta; tästä edettiin analysoimaan menetelmiä, joilla
haittaohjelmien tekijät voivat harhauttaa sumein tiivistein
tehtävää torjuntatyötä.

Lopuksi käsiteltiin rajoitteita, joita sumeat tiivisteet
tuovat mukanaan. Yleiseten rajoitteiden lisäksi
käesiteltiin Ssdeepin, Sdhashin ja Mvhash-b:n
toteutusten puutteita ja hyökkäysalttiutta, sekä
verrattiin toteutusten eroavaisuuksien ilmenemistä
käytännössä. Läpi käytiin myös sumeiden tiivisteiden
haavoittuvuuksia sekä näitä hyödyntäviä potentiaalisia
hyökkäyksiä.
